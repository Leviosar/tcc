\chapter{Discussão de Resultados}

Retornando ao problema dos prisioneiros apresentado no Capítulo 2 após termos apresentado tanto a técnica quanto sua implementação nos capítulos anteriores é possível realizar uma avaliação qualitativa dos resultados obtidos. Primeiramente podemos identificar que a implementação corresponde as prerrogativas do enunciado do problema, considerando que em uma comunicação a partir da metodologia proposta é utilizada uma chave privada definida previamente entre as duas partes, um objeto de cobertura arbitrário e se gera um estego objeto que pode ser trafegado em canal público. Devido à propriedade pseudoaleatória tanto dos bits gerados pela mensagem quanto operandos naturalmente codificados por instruções compiladas, é seguro afirmar que o carcereiro não teria a capacidade de identificar um subtexto malicioso dentro do estego objeto, e como assumimos a priori que a vigilância seria passiva, a destruição da integridade da mensagem não é preocupação deste estudo.

Existem ainda ataques comuns a classe de técnica utilizada na implementação deste trabalho descritos na litetura, que buscam encontrar assinaturas ou características comuns à transformações de código aplicadas \cite{ICISC04}. Um atacante buscando detectar essas assinaturas pode buscar pela presença de instruções pouco utilizadas em uma arquitetura a fim de identificar um padrão de substituição. Entretanto como o Algoritmo \ref{alg:one} não substitui instruções, trabalhando apenas com substituição de operandos, não há risco de ter essa propriedade explorada. Outra possibilidade é a detecção de uma frequência anormal de um certo tipo de instruções, também descartada pelo motivo anterior. Por fim, o atacante pode inspecionar o comportamento dos \textit{jumps} entre endereços do programa, buscando valores discrepantes que não seria naturalmente endereçados por compiladores. Tal vulnerabilidade também é inefetiva contra a aplicação desenvolvida pela ausência da alteração de valores imediatos em instruções de controle.

Nota-se então que os ataques já observados na bibliografia disponível não tem a capacidade de detectar as transformações realizadas pelo algoritmo proposto, o que não significa dizer que o mesmo seja indetectável, considerando que os ataques foram planejados para expor vulnerabilidades em técnicas aplicadas na arquitetura \textit{x86} que possui uma gama de possibilidades de alteração superior. Em fato, a eficiência de codificação de instruções da arquitetura RISC-V simultâneamente favorece a avaliação qualitativa de transformações binárias enquanto desfavorece uma avaliação quantitativa, reduzindo o número de permutações equivalentes dentro do programa e tornando a codificação de informações mais silenciosa porém menos eficiente.