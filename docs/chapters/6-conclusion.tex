% ----------------------------------------------------------
\chapter{Conclusão}
% ----------------------------------------------------------

O campo da esteganografia evolui pela história humana antes mesmo do cunho de seu próprio termo, entretanto suas aplicações com utilização voltada para formatos de código compilado ainda não possuem técnicas tão difundidas em comparação com outras categorias de objetos de cobertura, tornando a própria execução deste trabalho um desafio considerável. Após revisar e analisar quase 30 anos de estudos formais na área foi possível utilizar os conhecimentos e técnicas obtidas para aplicar uma metodologia de codificação de dados já proposta em uma arquitetura ainda não explorada com sucesso.

O algoritmo proposto é capaz de explorar redundâncias provenientes de operações comutativas na arquitetura RISC-V para codificar um bit a cada operação suportada e sua implementação foi desenvolvida de forma a minimizar o uso de recursos para possibilitar a execução eficiente mesmo em arquivos de grande porte. Apesar do resultado positivo na viabilidade da aplicação é necessário ressaltar que sua eficácia é reduzida em comparação com arquiteturas CISC devido a sua própria filosofia de elaboração, fornecendo um espaço útil de codificação com ao menos um nível de grandeza inferior. 

Ao avaliar a técnica por outro prisma, relevando a baixa taxa de codificação e focando primariamente na detectabilidade e corretude quanto ao problema dos prisioneiros, entende-se que os pontos negativos podem ser mitigados com o uso do algoritmo em situações específicas, aproveitando-se da dificuldade em revelar a mensagem sem conhecimento da chave para inserir mensagens de tamanho reduzido em arquivos. Tais situações podem ser exploradas para embutir assinaturas e identificadores em \textit{softwares} distribuídos comercialmente com o objetivo de rastrear violações de direitos autorais no combate à pirataria.

\section{Trabalhos Futuros}

A execução deste trabalho revela algumas portas em aberto para a evolução das técnicas de esteganografia em código compilado, principalmente considerando a arquitetura RISC-V. Durante a revisão bibliográfica foram encontrados pontos de exploração com potencial para codificação de dados, que poderiam ser unidos a técnica desenvolvida para incrementar a taxa de codificação. Inicialmente existe a possibilidade de explorar as instruções \texttt{HINT}, reservadas no espaço de endereçamento da arquitetura como livres para uso de implementações, normalmente utilizadas para otimizações de performance mas que poderiam ser subvertidas para carregar bits de mensagens. Além disso, no decorrer da pesquisa realizada o foco foi aplicar transformações diretamente no conjunto de instruções base da arquitetura, deixando de fora diversas extensões de propósitos específicos que apesar de possuírem frequência menor de aparecimento em um programa arbirtrário podem aumentar a taxa total de codificação de dados.

Ainda no escopo da arquitetura RISC-V e da implementação atual, levando em consideração que a validação de detectabilidade da técnica foi realizada a partir de ataques comuns direcionados a binários compilados para \textit{x86}, é necessário expo-la a técnica à uma gama maior de ataques que possam expor vulnerabilidades específicas da arquitetura para reforçar a confiabilidade do algoritmo.

Por fim, existem outras arquiteturas RISC amplamente utilizadas em dispositivos ao redor do mundo, principalmente da família ARM nas quais os conhecimentos adquiridos podem ser aplicados, resultando na portabilidade da técnica para um leque maior de usuários e possivelmente em resultados superiores se considerarmos que o endereçamento do RISC-V possui uma maior rigidez aos princípios RISC.