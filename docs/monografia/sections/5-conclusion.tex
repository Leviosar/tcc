\chapter{Considerações Finais}

O aprendizado da área de segurança computacional pode ser muito complicado, por isso ambientes para testes de vulnerabilidades são tão importantes. Com esses ambientes pode-se testar técnicas, correções, ferramentas, scripts, entre tantas outras coisas. Por este motivo este trabalho está sendo desenvolvido aspirando facilitar o percurso inicial na área. Além disso, ambientes em formato de imagem para Máquinas Virtuais já existem em certa quantidade, mas ambientes preparados para rodar em Docker e com isso economizar recursos são mais escassos.

Para este ambiente foram inicialmente escolhidas 7 vulnerabilidades que ainda podem ser encontradas com certa facilidade em sistemas web pelo mundo. Será necessário realizar testes para confirmar se o ambiente fica realmente segregado a fim de que máquinas usadas para testes não sejam vítimas de ataques enquanto usam o ambiente proposto.

\section{Cronograma}

\begin{center}
\begin{table}[!htb]
\begin{tabular}{|l|l|l|l|l|l|}
\hline
Etapa                                  & Março                  & Abril                  & Maio                   & Junho                  & Julho                  \\ \hline
Desenvolvimento do Back-end em NodeJS  & \multicolumn{1}{c|}{X} & \multicolumn{1}{c|}{X} &                        &                        &                        \\ \hline
Desenvolvilmento do Front-end em React & \multicolumn{1}{c|}{X} & \multicolumn{1}{c|}{X} &                        &                        &                        \\ \hline
Integração do Front-end com o Back-end &                        & \multicolumn{1}{c|}{X} & \multicolumn{1}{c|}{X} &                        &                        \\ \hline
Conclusão de escrita da monografia     &                        &                        & \multicolumn{1}{c|}{X} &                        &                        \\ \hline
Defesa de projeto                      &                        &                        &                        & \multicolumn{1}{c|}{X} &                        \\ \hline
Ajustes e envio final                  &                        &                        &                        &                        & \multicolumn{1}{c|}{X} \\ \hline
\end{tabular}
\end{table}
\end{center}