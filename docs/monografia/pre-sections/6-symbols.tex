%!TEX root = ../main.tex

\begin{listadesimbolos}

  $\gets$   & Atribuição \\
  $\exists$   & Quantificação existencial \\
  $\rightarrow$   & Implicação \\
  $\wedge$   & E lógico \\
  $\vee$   & Ou lógico \\
  $\neg$   & Negação lógica \\
  $\mapsto$   & Mapeia para \\
  $\sqsubseteq$   & Subclasse (em ontologias) \\
  $\subseteq$   & Subconjunto: $\forall x\;.\; x \in A \rightarrow x \in B$ \\
  $\langle\ldots\rangle$ & Tupla \\
  $\forall$   & Quantificação universal \\
  mmmmm & Nenhum sentido, apenas estou aqui para demonstrar a largura máxima dessas colunas. Ao abrir o ambiente \texttt{listadesimbolos}, pode-se fornecer um argumento opcional indicando a largura da coluna da esquerda (o default é de 5em): \texttt{\textbackslash{}begin\{listadesimbolos\}[2cm] .... \textbackslash{}end\{listadesimbolos\}} \\
  $\alpha$   & Alpha \\
  $\beta$   & Beta \\
  $\gamma$   & Gamma \\
  $\delta$   & Delta \\
  $\epsilon$   & Epsilon \\
  $\zeta$   & Zeta \\
  $\eta$   & Eta \\
  $\theta$   & Theta \\
  $\iota$   & Iota \\
  $\kappa$   & Kappa \\
  $\lambda$   & Lambda \\
  $\mu$   & Mu \\
  $\nu$   & Nu \\
  $\xi$   & Xi \\
  $\pi$   & Pi \\
  $\rho$   & Rho \\
  $\sigma$   & Sigma \\
  $\tau$   & Tau \\
  $\upsilon$   & Upsilon \\
  $\phi$   & Phi \\
  $\bowtie$  & Apertem os cintos, uma quebra de página se aproxima! \\
  $\oslash$   & Não use exclamações em lista de símbolos! \\
  $\varphi$   & Varphi \\
  $\chi$   & Chi \\
  $\psi$   & Psi \\
  $\omega$   & Omega \\

\end{listadesimbolos}