\def\msquare{\mathord{\scalerel*{\Box}{gX}}}
\newcommand\answerbox{%%
    \fbox{\rule{0.0001ex}{1pt}\rule[0.1ex]{0pt}{0.0001ex}}\hspace{5mm}
}


\begin{center}
    {\ABNTEXchapterfont\bfseries\Large FOLHA DE APROVAÇÃO DE PROPOSTA DO TCC}

  \end{center}
  
  \renewcommand\theadalign{cb}
  \renewcommand\theadfont{\bfseries}
	\renewcommand\theadgape{\Gape[4pt]}
	\renewcommand\cellgape{\Gape[4pt]}
	\begin{flushleft}
    \resizebox{\columnwidth}{!}{%
	\begin{tabular}{|l|l|}
	\hline
			\textbf{Acadêmico(s)} &  João Vitor Maia Neves Cordeiro \hspace{74mm} \\ \hline
            \textbf{Título do Trabalho}  & \makecell[l]{Desenvolvimento de uma técnica de esteganografia explorando arquivos binários de código compilado}\\ \hline
            \textbf{Curso} & Ciências da Computação/INE/UFSC \\ \hline
            \textbf{Área de Concentração} & \textbf{SEGURANÇA COMPUTACIONAL} \\ \hline
    \end{tabular}
    }
    
    \vspace{0.5cm}
    
    \textbf{Instruções para preenchimento pelo \emph{ORIENTADOR DO TRABALHO}:}
    \begin{itemize}
    \item Para cada critério avaliado, assinale um X na coluna SIM apenas se considerado aprovado. Caso contrário, indique as alterações necessárias na coluna de Observação.
    \end{itemize}
    \newcolumntype{C}{>{\centering\arraybackslash}p{30em}}
    \newcolumntype{A}{>{\centering\arraybackslash}p{4em}}
    \newcolumntype{O}{>{\centering\arraybackslash}p{11em}}
    \resizebox{\textwidth}{!}{%
    \begin{tabular}{|C|A|A|A|A|O|}
    \hline
    	\multirow{2}{*}{\cc \textbf{Critérios}} & \multicolumn{4}{c|}{\cc \textbf{Aprovado}} & \multirow{2}{*}{\cc \textbf{Observação}} \\[1ex]
	\cline{2-5}
        \cc & \cc \textbf{Sim} & \cc \textbf{Parcial} & \cc \textbf{Não} & \cc \textbf{Não se aplica} & \cc \\
        \hline
        \makecell[l]{O trabalho é adequado para um TCC em CCO \\ (relevância / abrangência)?}& X\cc & \cc & \cc & \cc & \\
        \hline
        \makecell[l]{O título é adequado?}& X\cc & \cc & \cc & \cc & \\
        \hline
        \makecell[l]{O Tema de pesquisa está claramente descrito?}& X\cc & \cc & \cc & \cc & \\
        \hline
        \makecell[l]{O problema/hipóteses de pesquisa do trabalho está \\ claramente identificado?}& X\cc & \cc & \cc & \cc & \\
        \hline
        \makecell[l]{A relevância da pesquisa é justificada?}& X\cc & \cc & \cc & \cc & \\
        \hline
        \makecell[l]{Os objetivos descrevem completa e claramente o que se\\ pretende alcançar neste trabalho?}& X\cc & \cc & \cc & \cc & \\
        \hline
        \makecell[l]{É definido o método a ser adotado no trabalho? O método\\ condiz com os objetivos e é adequado para um TCC?}& X\cc & \cc & \cc & \cc & \\
        \hline
        \makecell[l]{Foi definido um cronograma coerente com o método definido\\ (indicando todas as atividades) e com as datas das entregas\\ (p.ex. Projeto I, II, Defesa)?}& X\cc & \cc & \cc & \cc & \\
        \hline
        \makecell[l]{Foram identificados custos relativos à execução deste trabalho\\ (se houver)? Haverá financiamento para estes custos?}& \cc & \cc & \cc & X\cc & \\
        \hline
        \makecell[l]{Foram identificados todos os envolvidos neste trabalho?}& \cc & X\cc & \cc & \cc & \\
        \hline
        \makecell[l]{As formas de comunicação foram definidas?}& X\cc & \cc & \cc & \cc & \\[1ex]
        \hline
        \makecell[l]{Riscos potenciais que podem causar desvios do plano \\ foram identificados?}& X\cc & \cc & \cc & \cc & \\
        \hline
        \makecell[l]{Caso o TCC envolva a produção de um software ou outro tipo de \\produto e seja desenvolvido também como uma atividade\\ realizada numa empresa ou  laboratório, consta na proposta\\ uma declaração (Anexo 3) de ciência e concordância com a\\ entrega do código fonte e/ou documentação produzidos?}& \cc & \cc & \cc & X\cc & \\
        \hline
    \end{tabular}
    }
    \newcolumntype{B}{>{\centering\arraybackslash}p{41.72em}}
    \newcolumntype{E}{>{\centering\arraybackslash}p{7em}}

    \newcolumntype{H}{>{\centering\arraybackslash}p{41.72em}}

    \newcolumntype{F}{>{\centering\arraybackslash}p{15em}}
    \newcolumntype{G}{>{\centering\arraybackslash}p{10em}}


    \resizebox{1\textwidth}{!}{
  \begin{tabular}{|F|G|G|G|}
  \hline \textbf{Avaliação} & \multicolumn{2}{c}{ \makebox[1in]{} $\boxtimes$ Aprovado \makebox[1in]{ } $\Box$ Não Aprovado } & \\
  \hline
  \hline
  \textbf{Professor Responsável}: & Jean Everson Martina & \makebox[1in]{}18/07/2022 &\\
  \hline
  \textbf{Orientador Externo}: & & \makebox[1in]{} &\\
  \hline
  \end{tabular}
}
\end{flushleft}